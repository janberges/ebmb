\documentclass[a4paper]{article}

\usepackage[margin=35mm]{geometry}

\usepackage[utf8]{inputenc}
\usepackage[english]{babel}

\usepackage[math]{iwona}
\usepackage{inconsolata}
\usepackage[T1]{fontenc}

\usepackage[labelfont=bf]{caption}

\usepackage{mathtools, xcolor}

\usepackage[hidelinks]{hyperref}

\def\bottomfraction{0.9}

\def\D{\mathrm d}
\def\E{\mathrm e}
\def\I{\mathrm i}

\def\sub#1{\sb{\mathrm{#1}}}

\def\from#1{\sb{\mathrlap{#1}}}
\def\till#1{\sp{\mathrlap{#1}}}

\let\vec\boldsymbol

\let\Delta\varDelta
\let\epsilon\varepsilon
\let\Lambda\varLambda
\let\Sigma\varSigma
\let\Theta\varTheta

\newlength\gap
\setlength\gap{6pt}

\def\slant#1{\rlap{\rotatebox{45}{#1}}\vphantom{\rotatebox{45}{#1}}}

\def\stack#1#2#3{\begin{tabular}{@{}#1@{}}#2\\#3\end{tabular}}

\def\dummy#1{\textcolor{darkgray}{$\langle$#1$\rangle$}}
\def\dtype#1{\textcolor{darkgray}{\textsc{#1}}}

\def\headline#1{\section*{%
   \normalsize\normalfont%
   \rlap{\rule[0.5ex]\textwidth{0.4pt}}%
   \qquad\colorbox{white}{#1}%
   }}

\begin{document}
   \begin{center}
      \LARGE \texttt{ebmb} \par \bigskip
      \large Solve multiband \textsc{Eliashberg} equations
   \end{center}

   \headline{Outline}

   This software provides three programs:
   %
   \begin{enumerate}
      \item
         \texttt{ebmb} itself solves the multiband \textsc{Eliashberg} equations
         (Eqs.~\ref{Eliashberg equations} or \ref{cDOS Eliashberg equations}) on
         a cut-off imaginary axis and optionally continues the results to the
         real axis via \textsc{Padé} approximants.

         A material is defined by nothing but an \textsc{Eliashberg} spectral
         function or, as fallback, an \textsc{Einstein} phonon frequency and
         intra- and interband electron-phonon couplings, \textsc{Coulomb}
         pseudo-potentials and, if desired, the band densities of \textsc{Bloch}
         states, otherwise assumed to be constant.

      \item
         \texttt{critical} finds the critical point via the bisection method
         varying a parameter of choice. Superconductivity is defined by the
         kernel of the linearized gap equation (Eq.~\ref{linearized gap
         equation}) having an eigenvalue greater than or equal to unity.

      \item
         \texttt{tc} finds the critical temperature for each band separately via
         the bisection method. Superconductivity is defined by the order
         parameter exceeding a certain threshold. Usually, it is preferable to
         use \texttt{critical}.

   \end{enumerate}

   \headline{Installation}

   The makefile is designed for the \emph{GNU} or \emph{Intel} Fortran compiler:
   %
   \begin{quote}
      \verb|$ make FC=gfortran FFLAGS=-O3|
   \end{quote}

   \headline{I/O}

   \begin{itemize}
      \item
         Parameters are defined on the command line:
         %
         \begin{quote}
            \verb|$| \dummy{program}
               \dummy{key\,1}\verb|=|\dummy{value\,1}
               \dummy{key\,2}\verb|=|\dummy{value\,2} \verb|...|
         \end{quote}
         %
         The available keys and default values are listed in
         Table~\ref{parameters}.

      \item
         Unless \texttt{tell=false}, the results are printed to standard output.

      \item
         Unless \texttt{file=none}, a binary output file is created. For
         \texttt{critical} and \texttt{tc} it simply contains one or more double
         precision floating point numbers, for \texttt{ebmb} the format defined
         in Tables~\ref{output} and \ref{identifiers} is used.

      \item
         The provided \emph{Python} wrapper functions load the results into
         \emph{NumPy} arrays:
         %
         \begin{quote}
            \verb|import ebmb| \\
            \verb|results = ebmb.get(|%
            \dummy{program}\verb|, |%
            \dummy{file}\verb|, |%
            \dummy{replace}\verb|,| \\
            \verb|    |%
               \dummy{key\,1}\verb|=|\dummy{value\,1}\verb|, |%
               \dummy{key\,2}\verb|=|\dummy{value\,2}\verb|, ...|\verb|)|
         \end{quote}
         %
         \dummy{replace} decides whether an existing \dummy{file} is used or
         overwritten.

         Given a band structure, its discretized domain and $n - 1$ filters, an
         input file with the density of states resolved for $n$ subdomains is
         generated like this:
         %
         \begin{quote}
            \verb|from numpy import cos, dot, linspace, pi| \\
            \verb|DOSfile('dos.in', epsilon=lambda *k: -cos(k).sum() / 2, | \\
            \verb|    domain=[linspace(-pi, pi, 1000, endpoint=False)] * 2,| \\
            \verb|    filters=[lambda *k: pi ** 2 / 2 <= dot(k, k) <= pi ** 2])|
         \end{quote}
   \end{itemize}

   \headline{\textsc{Eliashberg} theory}

   Let $\hbar = k \sub B = 1$. Fermionic and bosonic \textsc{Matsubara}
   frequencies are defined as $\omega_n = (2 n + 1) \pi T$ and $\nu_n = 2 n \pi
   T$, respectively. The quantity of interest is the \textsc{Nambu} self-energy
   matrix%
   %
   \footnote{Y. \textsc{Nambu}, Phys. Rev. \textbf{117}, 648 (1960)}
   %
   \begin{equation*}
      \vec \Sigma_i(n)
      = \I \omega_n [1 - Z_i(n)] \vec 1
      + \underbrace{Z_i(n) \, \Delta_i(n)}
      _ {\displaystyle \phi_i(n)} \vec \sigma_1
      + \chi_i(n) \vec \sigma_3,
   \end{equation*}
   %
   where the \textsc{Pauli} matrices are defined as usual and $i$ is a band
   index. Renormalization $Z_i(n)$, order parameter $\phi_i(n)$ and energy shift
   $\chi_i(n)$ are determined by the \textsc{Eliashberg} equations%
   %
   \footnote{%
      G. M. \textsc{Eliashberg}, Soviet Phys. JETP \textbf{11}, 696 (1960).
      \newline
      A comprehensive review is given by P. B. \textsc{Allen} and B.
      \textsc{Mitrović} in Solid state physics \textbf{37} (1982)
      }
   %
   \begin{equation}
      \begin{split}
         Z_i(n) &= 1 + \frac T {\omega_n} \sum_j \sum_{m = 0}^{N - 1}
         \int \from{-\infty} \till \infty \D \epsilon
         \frac{n_j(\epsilon)}{n_j(\mu_0)}
         \frac{\omega_m Z_j (m)}{\Theta_j(\epsilon, m)}
         \Lambda_{i j}^-(n, m),
         \\
         \phi_i(n) &= T \sum_j \sum_{m = 0}^{N - 1}
         \int \from{-\infty} \till \infty \D \epsilon
         \frac{n_j(\epsilon)}{n_j(\mu_0)}
         \frac{\phi_j(m)}{\Theta_j(\epsilon, m)}
         [\Lambda_{i j}^+(n, m) - U^*_{i j}(m)],
         \\
         \chi_i(n) &= -T \sum_j \sum_{m = 0}^{N - 1}
         \int \from{-\infty} \till \infty \D \epsilon
         \frac{n_j(\epsilon)}{n_j(\mu_0)}
         \frac{\epsilon - \mu + \chi_j(m)}{\Theta_j(\epsilon, m)}
         \Lambda_{i j}^+(n, m),
         \\
         \Theta_i(\epsilon, n) &=
         [\omega_n Z_i(n)]^2 + \phi_i^2(n) + [\epsilon - \mu + \chi_i(n)]^2,
      \end{split}
      \label{Eliashberg equations}
   \end{equation}
   %
   and may then be analytically continued to the real-axis by means of
   \textsc{Padé} approximants.%
   %
   \footnote{%
      H. J. \textsc{Vidberg} and J. W. \textsc{Serene}, J. Low Temp. Phys.
      \textbf{29}, 179 (1977)
      }
   %
   The electron-phonon coupling matrices and the rescaled \textsc{Coulomb}
   pseudo-potential are connected to the corresponding input parameters via
   %
   \begin{align}
      \Lambda_{i j}^\pm(n, m) &=
      \lambda_{i j}(n - m) \pm \lambda_{i j}(n + m + 1),
      &
      \lambda_{i j}(n) &=
      \int \from 0 \till \infty \D \omega
      \frac{2 \omega \, \alpha^2 F_{i j}(\omega)}{\omega^2 + \nu_n^2}
      \underset {\underset{\text{Einstein}}\uparrow} =
      \frac{\lambda_{i j}}{1 + \big[ \frac{\nu_n}{\omega \sub E} \big]^2},
      \notag
      \\
      U^*_{i j}(m) &=
      \begin{cases}
         2 \mu^*_{i j}(N \sub C) & \text{for $m < N \sub C$,} \\
         0                       & \text{otherwise,}
      \end{cases}
      &
      \frac 1 {\mu^*_{i j}(N \sub C)} &=
      \frac 1 {\mu^*_{i j}} + \ln \frac{\omega \sub E}{\omega_{N \sub C}}
      \label{cDOS rescaling}
   \end{align}
   %
   with the \textsc{Eliashberg} spectral function $\alpha^2 F_{i j}(\omega)$.
   Alternatively, if the band density $n_i(\epsilon)$ of \textsc{Bloch} states
   with energy $\epsilon$ per spin, band and unit cell is given,
   %
   \begin{equation}
       \frac 1 {\mu^*_{i j}(N \sub C)} =
       \frac 1 {\mu^*_{i j}} + \ln \frac {2 \omega \sub E} D +
       \frac 1 \pi \sum_i \int \from{-\infty} \till \infty \D \epsilon
       \frac{n_i(\epsilon)}{n_i(\mu_0)}
       \begin{cases}
           \frac 1 {\epsilon - \mu_0}
           \arctan \frac{\epsilon - \mu_0}{\omega_{N \sub C}}
               & \text{for $\epsilon \neq \mu_0$,} \\
           \frac 1 {\omega_{N \sub C}}
               & \text{otherwise,}
       \end{cases}
      \label{rescaling}
   \end{equation}
   %
   where $D$ is the electronic bandwidth. $\mu_0$ and $\mu$ are the chemical
   potentials for free and interacting particles, respectively. The latter
   ensures that the particle number is conserved:
   %
   \begin{equation*}
      2 \sum_i \int \from{-\infty} \till \infty \D \epsilon
      \frac{n_i(\epsilon)}{\E^{(\epsilon - \mu_0) / T} + 1}
      = n_0 \overset ! = n
      \approx 1 - 4 T \sum_i \int \from{-\infty} \till \infty \D \epsilon \,
      n_i(\epsilon)
      \Bigg[
         \sum_{n = 0}^{N - 1}
         \frac{\epsilon - \mu + \chi_i(n)}{\Theta_i(\epsilon, n)}
         + \frac{\arctan \frac{\epsilon - \mu}{\omega_N}}{2 \pi T}
      \Bigg].
   \end{equation*}
   %
   Approximating $n_i(\epsilon) \approx n_i(\mu_0)$ yields $\chi_i(n) = 0$ and
   the constant-DOS \textsc{Eliashberg} equations
   %
   \begin{equation}
      \begin{split}
         Z_i(n) &= 1 + \frac{\pi T}{\omega_n} \sum_j \sum_{m = 0}^{N - 1}
         \frac{\omega_m}{\sqrt{\omega_m^2 + \Delta_j^2(m)}}
         \Lambda_{i j}^-(n, m),
         \\
         \Delta_i(n) &= \frac{\pi T}{Z(n)} \sum_j \sum_{m = 0}^{N - 1}
         \frac{\Delta_j(m)}{\sqrt{\omega_m^2 + \Delta_j^2(m)}}
         [\Lambda_{i j}^+(n, m) - U^*_{i j}(m)].
      \end{split}
      \label{cDOS Eliashberg equations}
   \end{equation}
   %
   At the critical temperature, $\Delta_i(m)$ is infinitesimal and negligible
   relative to $\omega_m$. This yields
   %
   \begin{equation}
      \begin{split}
         \Delta_i(n) &= \sum_j \sum_{m = 0}^{N - 1}
         K_{i j}(n, m) \, \Delta_j(m),
         \\
         K_{i j}(n, m) &= \frac 1 {2 m + 1} [\Lambda_{i j}^+(n, m)
         - \delta_{i j} \delta_{n m} D_i^N(n) - U^*_{i j}(m)],
         \\
         D_i^N(n) &= \sum_j \sum_{m = 0}^{N - 1} \Lambda_{i j}^-(n, m)
         \overset{N = \infty} =
         \sum_j \Big[ \lambda_{i j} + 2 \sum_{m = 1}^n \lambda_{i j}(m) \Big].
      \end{split}
      \label{linearized gap equation}
   \end{equation}
   %
   $Z_i(n)$ is not biased by the cutoff if $D_i^\infty(n)$ is used in place of
   $D_i^N(n)$ in the kernel $K_{i j}(n, m)$.

   For a given scalar $\alpha^2 F(\omega)$, an effective phonon frequency can be
   calculated in different ways. We follow \textsc{Allen} and \textsc{Dynes},%
   %
   \footnote{%
      P. B. \textsc{Allen} and R. C. \textsc{Dynes}, Phys. Rev. B \textbf{12},
      905 (1975)
      }
   %
   who define the logarithmic average frequency
   %
   \begin{equation*}
       \omega \sub{log} = \exp
       \Bigg[
          \frac 2 \lambda \int_0^\infty
          \frac{\D \omega} \omega
          \alpha^2 F(\omega) \ln(\omega)
       \Bigg]
   \end{equation*}
   %
   and the second-moment average frequency
   %
   \begin{equation*}
       \overline \omega_2 = \sqrt
       {
          \frac 2 \lambda \int_0^\infty
          \D \omega \,
          \alpha^2 F(\omega) \, \omega
       }.
   \end{equation*}
   %
   and choose $\overline \omega_2$ for $\omega \sub E$ in Eqs.~\ref{cDOS
   rescaling} and \ref{rescaling} for rescaling $\mu^*$.

   \headline{Acknowledgment}

   Parts of the program are inspired by the EPW code%
   %
   \footnote{%
      See F. \textsc{Giustino}, M. L. \textsc{Cohen} and S. G. \textsc{Louie},
      Phys.  Rev. B \textbf{76}, 165108 (2007) for a methodology review.
      \newline
      Results related to \textsc{Eliashberg} theory are given by E. R.
      \textsc{Margine} and F. \textsc{Giustino}, Phys. Rev. B \textbf{87},
      024505 (2013)
      }
   %
   and work of Malte Rösner.

   \headline{Contact}

   Any feedback may be directed to \href
      {mailto:jan.berges@uni-bremen.de?subject=ebmb}
      {\tt jan.berges@uni-bremen.de}.

   \newgeometry{top=25mm, bottom=25mm}

   \begin{table}[b]
      \centering
      \begin{tabular}{*9l}
         \bf key           & \bf default & \bf \slant{unit} & \bf \slant{symbol}  & \bf description & \tt \slant{ebmb} & \tt \slant{tc} & \tt \slant{critical} & \bf \slant{variable} \\
         \tt file          & \tt none    & --               & --                  & output file                                & $+$ & $+$ & $+$ & $-$ \\
         \tt form          & \tt F16.12  & --               & --                  & number edit descriptor                     & $+$ & $+$ & $+$ & $-$ \\
         \tt tell          & \tt true    & --               & --                  & use standard output?                       & $+$ & $+$ & $+$ & $-$ \\[\gap]
         \tt T             & 10          & K                & $T$                 & temperature                                & $+$ & $+$ & $+$ & $+$ \\[\gap]
         \tt omegaE        & 0.02        & eV               & $\omega \sub E$     & \textsc{Einstein} frequency                & $+$ & $+$ & $+$ & $+$ \\
         \tt cutoff        & 15          & $\omega \sub E$  & $\omega_N$          & overall cutoff frequency                   & $+$ & $+$ & $+$ & $-$ \\
         \tt cutoffC       & $\omega_N$  & $\omega \sub E$  & $\omega_{N \sub C}$ & \textsc{Coulomb} cutoff frequency          & $+$ & $+$ & $+$ & $-$ \\[\gap]
         \tt lambda, lamda & 1           & 1                & $\lambda_{i j}$     & electron-phonon coupling                   & $+$ & $+$ & $+$ & $+$ \\
         \tt muStar, mu*   & 0           & 1                & $\mu^*_{i j}$       & rescaled \textsc{Coulomb} potential        & $+$ & $+$ & $+$ & $+$ \\
         \tt muC           & 0           & 1                & $\mu_{i j}$         & unscaled \textsc{Coulomb} parameter        & $+$ & $+$ & $+$ & $+$ \\[\gap]
         \tt bands         & 1           & 1                & --                  & number of bands                            & $+$ & $+$ & $+$ & $-$ \\[\gap]
         \tt dos, DOS      & \tt none    & --               & --                  & file with density of states                & $+$ & $+$ & $+$ & $-$ \\
         \tt a2f, a2F      & \tt none    & --               & --                  & file with \textsc{Eliashberg} function     & $+$ & $+$ & $+$ & $-$ \\[\gap]
         \tt n             & --          & 1                & $n_0$               & initial occupancy number                   & $+$ & $+$ & $+$ & $-$ \\
         \tt mu            & 0           & eV               & $\mu_0$             & initial chemical potential                 & $+$ & $+$ & $+$ & $-$ \\
         \tt conserve      & \tt true    & --               & --                  & conserve particle number?                  & $+$ & $+$ & $+$ & $-$ \\
         \tt chi           & \tt true    & --               & --                  & consider energy shift?                     & $+$ & $+$ & $+$ & $-$ \\[\gap]
         \tt limit         & 250000      & 1                & --                  & maximum number of iterations               & $+$ & $+$ & $+$ & $-$ \\[\gap]
         \tt epsilon       & $10^{-13}$  & a.u.             & --                  & negligible float difference                & $+$ & $+$ & $+$ & $-$ \\
         \tt error         & $10^{-5}$   & a.u.             & --                  & bisection error                            & $-$ & $+$ & $+$ & $-$ \\
         \tt zero          & $10^{-10}$  & eV               & --                  & negligible gap at $T \sub c$ (threshold)   & $-$ & $+$ & $-$ & $-$ \\
         \tt rate          & $10^{-1}$   & 1                & --                  & growth rate for bound search               & $-$ & $+$ & $+$ & $-$ \\[\gap]
         \tt lower         & 0           & eV               & --                  & minimum real-axis frequency                & $+$ & $-$ & $-$ & $-$ \\
         \tt upper         & --          & eV               & --                  & maximum real-axis frequency                & $+$ & $-$ & $-$ & $-$ \\
         \tt clip          & 15          & $\omega \sub E$  & --                  & maximum real-axis frequency                & $+$ & $-$ & $-$ & $-$ \\
         \tt eta, 0+       & 0           & eV               & --                  & infinitesimal energy $0^+$                 & $+$ & $-$ & $-$ & $-$ \\
         \tt resolution    & 0           & 1                & --                  & resolution of real-axis solution           & $+$ & $-$ & $-$ & $-$ \\
         \tt measurable    & \tt false   & --               & --                  & find measurable gap?                       & $+$ & $-$ & $-$ & $-$ \\[\gap]
         \tt unscale       & \tt true    & --               & --                  & estimate missing {\tt muC} from {\tt mu*}? & $+$ & $+$ & $+$ & $-$ \\
         \tt rescale       & \tt true    & --               & --                  & use $\mu^*_{i j}$ rescaled for cutoff?     & $+$ & $+$ & $+$ & $-$ \\
         \tt imitate       & \tt false   & --               & --                  & use $Z_i(n)$ biased by cutoff?             & $-$ & $-$ & $+$ & $-$ \\[\gap]
         \tt normal        & \tt false   & --               & --                  & enforce normal state?                      & $+$ & $-$ & $-$ & $-$ \\[\gap]
         \tt power         & \tt true    & --               & --                  & power method for single band?              & $-$ & $-$ & $+$ & $-$
      \end{tabular}
      \captionsetup{singlelinecheck=off}
      \caption[Input parameters]{
         Input parameters.

         \begin{itemize}
            \item
               The columns \texttt{ebmb}, \texttt{tc} and \texttt{critical} show
               which keys are used by these programs.

            \item
               The rightmost column indicates which parameters may be chosen as
               variable for \texttt{critical}. The variable is marked with a
               negative sign; its absolute value is used as initial guess. If no
               parameter is negative, the critical temperature is searched for.

            \item
               \texttt{lambda}, \texttt{muStar}, and \texttt{muC} expect
               flattened square matrices of equal size the elements of which
               are separated by commas. It is impossible to vary more than one
               element at once.

            \item
               \texttt{dos} has lines $\epsilon$/eV $n_1$/a.u. $n_2$/a.u.
               $\dots$ with $\epsilon$ ascending but not necessarily
               equidistant.

            \item
               \texttt{a2F} has lines $\omega$/eV $\alpha^2 F_{1, 1}$ $\alpha^2
               F_{2, 1}$ $\dots$ with $\omega$ ascending but not necessarily
               equidistant.
         \end{itemize}
         }
      \label{parameters}
   \end{table}

   \begin{table}
      \centering
      \begin{tabular}{r l}
         \dummy{\dtype{characters} key}\verb|:|\rlap{%
         \dummy{$n_1 \times \hdots \times n_r$ \dtype{numbers} value}}
                      & \\
                      & associate key with value \\[\gap]
         \verb|DIM:|\rlap{%
         \dummy{\dtype{integer} $r$}%
         \dummy{$r$ \dtype{integers} $n_1 \dots n_r$}}
                      & \\
                      & define shape (column-major) \\[\gap]
         \verb|INT:|  & take \dtype{numbers} as \dtype{integers} \\[\gap]
         \verb|REAL:| & take \dtype{numbers} as \dtype{doubles}
      \end{tabular}
      \captionsetup{width=0.5\textwidth}
      \caption{
         Statements allowed in binary output. The data types \dtype{character},
         \dtype{integer} and \dtype{double} take 1, 4 and 8 bytes of storage,
         respectively.
         }
      \label{output}
   \end{table}

   \begin{table}
      \centering
      \begin{tabular}{*2l r}
         \hline
         \multicolumn3l{\textbf{imaginary-axis results} \hfill} \\
         \hline
         \tt iomega & \textsc{Matsubara} frequency (without $\I$) & $\omega_n$        \\
         \tt Delta  & gap                                         & $\Delta_i(n)$     \\
         \tt Z      & renormalization                             & $Z_i(n)$          \\
         \tt chi    & energy shift                                & $\chi_i(n)$       \\
         \tt phiC   & constant \textsc{Coulomb} contribution      & ${\phi \sub C}_i$ \\
         \tt status & status (steps till convergence or $-1$)     & --                \\
         \hline
         \multicolumn3l{\textbf{occupancy results}} \\
         \hline
         \tt \stack{l}{n0} {n}  & \stack{r}{initial}{final} $\Big\}$ occupancy number   & \stack{r}{$n_0$}  {$n$}   \\
         \tt \stack{l}{mu0}{mu} & \stack{r}{initial}{final} $\Big\}$ chemical potential & \stack{r}{$\mu_0$}{$\mu$} \\
         \hline
         \multicolumn3l{\textbf{effective parameters} \hfill \texttt{a2F given}} \\
         \hline
         \tt lambda   & electron-phonon coupling        & $\lambda_{i j}$      \\
         \tt omegaE   & \textsc{Einstein} frequency     & $\omega \sub E$      \\
         \tt omegaLog & logarithmic average frequency   & $\omega \sub{log}$   \\
         \tt omega2nd & second-moment average frequency & $\overline \omega_2$ \\
         \hline
         \multicolumn3l{\textbf{real-axis results} \hfill \texttt{resolution} > 0} \\
         \hline
         \tt omega                           & frequency                                           & $\omega$           \\
         \tt \stack{l}{Re[Delta]}{Im[Delta]} & \stack{r}{real}{imaginary} $\Big\}$ gap             & $\Delta_i(\omega)$ \\
         \tt \stack{l}{Re[Z]}    {Im[Z]}     & \stack{r}{real}{imaginary} $\Big\}$ renormalization & $Z_i(\omega)$      \\
         \tt \stack{l}{Re[chi]}  {Im[chi]}   & \stack{r}{real}{imaginary} $\Big\}$ energy shift    & $\chi_i(\omega)$   \\
         \hline
         \multicolumn3l{\textbf{measurable results} \hfill \texttt{measurable=true}} \\
         \hline
         \tt Delta0  & measurable gap           & \llap{${\Delta_0}_i = \operatorname{Re}[\Delta_i({\Delta_0}_i)]$} \\
         \tt status0 & status of measurable gap & --
      \end{tabular}
      \caption{Keys used in binary output.}
      \label{identifiers}
   \end{table}
\end{document}
